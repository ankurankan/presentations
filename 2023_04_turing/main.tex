\documentclass{beamer}

\usepackage[utf8]{inputenc}
\usecolortheme{beaver}
\usepackage{caption}
\usepackage{subcaption}
\usepackage{mathtools}
\usepackage{todonotes}
\usepackage{amsmath}
\usepackage{bm}
\usepackage{listings}
\usepackage{ragged2e}
\usepackage{fancyvrb}
\usepackage{titlecaps}

\Addlcwords{for a is but and with of in as the etc on to if}

\def\ci{\perp\!\!\!\!\!\perp}

\tikzstyle{latent} = [ draw, circle, inner sep = 2pt, minimum size = 0.65cm ]
\tikzstyle{observed} = [ draw, rectangle, inner sep = 2pt, minimum size = 0.65cm ]

\newtheorem{proposition}{Proposition}

\setbeamertemplate{section in toc}{\inserttocsectionnumber.~\inserttocsection}
\usetheme{Boadilla}
\makeatletter
\setbeamertemplate{footline}{%
    \leavevmode%
    \hbox{%
        \begin{beamercolorbox}[wd=.3\paperwidth,ht=2.25ex,dp=1ex,center]{author in head/foot}%
            \usebeamerfont{author in head/foot}\insertshortauthor\expandafter\beamer@ifempty\expandafter{\beamer@shortinstitute}{}{~~(\insertshortinstitute)}
        \end{beamercolorbox}%
        \begin{beamercolorbox}[wd=.55\paperwidth,ht=2.25ex,dp=1ex,center]{title in head/foot}%
            \usebeamerfont{title in head/foot}\insertshorttitle
        \end{beamercolorbox}%
        \begin{beamercolorbox}[wd=.15\paperwidth,ht=2.25ex,dp=1ex,right]{date in head/foot}%
            \usebeamerfont{date in head/foot}\insertshortdate{}\hspace*{2em}
            \insertframenumber{} / \inserttotalframenumber\hspace*{2ex} 
        \end{beamercolorbox}}%
        \vskip0pt%
    }
\makeatother

\begin{document}

\title[]{Papers}
\author [] {Ankur Ankan}
\date{}
\maketitle

\begin{frame}
	\frametitle{\titlecap{Conditional Independence (CI) Test for combination of discrete, ordinal, and continuous variables}}
	\begin{figure}
		\centering
		\includegraphics[scale=0.4]{imgs/ankan_textor}
		\caption*{}
	\end{figure}
	\vspace{-2em}
	\begin{itemize}
		\item Paper only considers ordinal and categorical variables
			but already working on an extension to continuous
			variables.
	\end{itemize}
\end{frame}

\begin{frame}
	\frametitle{\titlecap{CI Test: Main Idea}}
	\begin{enumerate}
		\item A residualization based approach. Main idea behind
			residualization tests are to test whether the variables
			are independent once the effect of the conditional
			variable is removed.
		\item Already there for continuous variables. E.g. Partial correlation test uses a regression equation for prediction.
		\item Extended to categorical and ordinal variables but no easy way to define residuals. 
		\item We use LS-Residuals which has theoretical properties which allows us to use it for this approach.
	\end{enumerate}
\end{frame}

\begin{frame}
	\frametitle{\titlecap{CI Test: Algorithm}}
	A residualization based approach. Given a dataset $ \mathbf{D} $ and a CI statement $ X \ci Y | \bm{Z} $:
	\begin{enumerate}
		\item If either $ X $ or $ Y $ are non-binary categorical variables,
			dummy encode them.
		\item Train two probability estimators $ p_x = \bm{x} \sim \bm{z} $ and
			$ p_y = \bm{y} \sim \bm{z} $.
			\begin{itemize}
				\item Any estimator that gives valid residuals can be used.
			\end{itemize}
		\item Make probability predictions using these two estimators 
			$ \hat{p}(\bm{x}) = p_x(\bm{z}) $ and $ \hat{p}(\bm{y}) = p_y(\bm{\bm{z}}) $.
		\item Use $ \hat{p}(\bm{x}) $, $ \hat{p}(\bm{y}) $, and $ D $ to compute LS-Residuals $ R_{\bm{x}|\bm{z}} $ and $ R_{\bm{y}|\bm{z}} $.	
			$$ R_{y_i | z_i} = \hat{p}(Y < y_i | Z=z_i) - \hat{p}(Y>y_i|Z=z_i) $$
		\item Compute test statistic and df using $ R_{\bm{x}|\bm{z}} $ and $ R_{\bm{y}|\bm{z}} $. The test statistic is $ \chi^2 $ distributed under the null.
			Define LS-Residual
	\end{enumerate}
\end{frame}

\begin{frame}
	\frametitle{\titlecap{CI Test: Possible Applications}}
	\begin{itemize}
		\item Constraint-Based Structure Learning Algorithms: Runs CI tests on the dataset to construct the model structure.
		\item Edge weights for model visualization
	\end{itemize}
\end{frame}

\begin{frame}
	\frametitle{\titlecap{MAG and PAG}}
\end{frame}

\begin{frame}
	\frametitle{\titlecap{Learning from interventional datasets}}
\end{frame}

\begin{frame}
	\frametitle{\titlecap{New data structure for faster inference}}
\end{frame}
\end{document}
