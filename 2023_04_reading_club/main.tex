\documentclass{beamer}

\usepackage[utf8]{inputenc}
\usecolortheme{beaver}
\usepackage{caption}
\usepackage{subcaption}
\usepackage{mathtools}
\usepackage{todonotes}
\usepackage{amsmath}
\usepackage{bm}
\usepackage{listings}
\usepackage{ragged2e}
\usepackage{fancyvrb}
\usepackage{titlecaps}

\Addlcwords{for a is but and with of in as the etc on to if}

\def\ci{\perp\!\!\!\!\!\perp}

\tikzstyle{latent} = [ draw, circle, inner sep = 2pt, minimum size = 0.65cm ]
\tikzstyle{observed} = [ draw, rectangle, inner sep = 2pt, minimum size = 0.65cm ]

\newtheorem{proposition}{Proposition}

\setbeamertemplate{section in toc}{\inserttocsectionnumber.~\inserttocsection}
\usetheme{Boadilla}
\makeatletter
\setbeamertemplate{footline}{%
    \leavevmode%
    \hbox{%
        \begin{beamercolorbox}[wd=.3\paperwidth,ht=2.25ex,dp=1ex,center]{author in head/foot}%
            \usebeamerfont{author in head/foot}\insertshortauthor\expandafter\beamer@ifempty\expandafter{\beamer@shortinstitute}{}{~~(\insertshortinstitute)}
        \end{beamercolorbox}%
        \begin{beamercolorbox}[wd=.55\paperwidth,ht=2.25ex,dp=1ex,center]{title in head/foot}%
            \usebeamerfont{title in head/foot}\insertshorttitle
        \end{beamercolorbox}%
        \begin{beamercolorbox}[wd=.15\paperwidth,ht=2.25ex,dp=1ex,right]{date in head/foot}%
            \usebeamerfont{date in head/foot}\insertshortdate{}\hspace*{2em}
            \insertframenumber{} / \inserttotalframenumber\hspace*{2ex} 
        \end{beamercolorbox}}%
        \vskip0pt%
    }
\makeatother

\begin{document}

\title[]{Validating Causal Inference Methods}
\author [] {}
\date{}

\begin{frame}
	\frametitle{Motivation}
	\begin{itemize}
		\item Many statistical methods for causal inference under unconfoundedness conditions. E.g. proposity score method, doubly robust, etc.
		\item These methods have only been evaluated on simualted data which might not be representative of real at hand.
		\item Crucial to understand how these methods perform on the data at hand.
	\end{itemize}
	\begin{figure}
		\includegraphics[scale=0.6]{imgs/comparison.png}
	\end{figure}
\end{frame}

\begin{frame}
	\frametitle{\titlecap{Proposed Framework}}
	\vspace{-2em}
	\begin{figure}
		\centering
		\includegraphics[scale=0.65]{imgs/flow.png}
	\end{figure}
	\vspace{-1.5em}
	\begin{itemize}
		\item A synthetic data generation approach.
		\item Uses user specified form and magnitude of causal effects
			and confounding bias.
		\item Trains a generative model that can generate
			stochastically indistinguishable dataset.
		\item Evaluate the causal methods by comparing their estimates
			with the simulated data estimates.
	\end{itemize}
\end{frame}

\begin{frame}
	\frametitle{\titlecap{Proposed Framework}}
	\begin{columns}
		\begin{column}{0.5 \textwidth}
			\begin{itemize}
				\item Generative model minimizes the metric: \todo[inline]{Insert the equation}
				\item For this trains two separate conditional VAEs: 1) $P(X \mid Z) $, 2) $ P(Y \mid X, Z) $.
				\item Use the simulated dataset for computing the true causal estimate.
			\end{itemize}
		\end{column}
		\begin{column}{0.5 \textwidth}
			\begin{figure}
				\includegraphics[scale=0.45]{imgs/distributions.png}
			\end{figure}
		\end{column}
	\end{columns}
\end{frame}

\begin{frame}
	\frametitle{\titlecap{Proposed Framework}}
% 	\begin{tikzpicture}	
% 		\tikzstyle{every node}=[align=center, inner sep=1pt]
% 		\node (x) at (0, 0) {$ X $};
% 		\node (epsx) at (0, 1) {$ \epsilon_X $};
% 		\node (z) at (1, 0) {$ Z $};
% 		\node (epsz) at (1, 1) {$ \epsilon_Z $};
% 		\node (u) at (2, 1) {$ U $};
% 		\node (y) at (3, 0) {$ Y $};
% 		\node (epsy) at (3, 1) {$ \epsilon_Y $};
% 
% 		\draw [->] (epsx) -- (x);
% 		\draw [->] (x) -- (z);
% 		\draw [->] (epsz) -- (z);
% 		\draw [->] (u) -- (z);
% 		\draw [->] (epsy) -- (y);
% 		\draw [->] (z) -- (y);
% 		\draw [->] (u) -- (y);
% 	\end{tikzpicture}
	Evaluation:
	Generating data from the trained model.
	Comparing it to the estimated estimates.
\end{frame}

\begin{frame}
	\frametitle{\titlecap{Discussion Points}}
	\begin{itemize}
		\item Multiple causal models can generate the same observational dataset. How realistic is the assumption that the causal effect estimate from the generative model is same as the true DGP.
		\item Why is specifying the treatment effect and confounding bias a desirable propoerty?
		\item What is the data that the VAE is getting trained on? What did they mean by training it on a set of global datasets?
	\end{itemize}
\end{frame}
\end{document}
