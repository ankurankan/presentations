\documentclass[aspectratio=169]{beamer}

\usepackage[utf8]{inputenc}
\usecolortheme{beaver}
\usepackage{caption}
\usepackage{subcaption}
\usepackage{mathtools}
\usepackage{todonotes}
\usepackage{amsmath}
\usepackage{bm}
\usepackage{listings}
\usepackage{ragged2e}
\usepackage{titlecaps}
\usepackage{fancyvrb}
\def\ci{\perp\!\!\!\!\!\perp}

\newtheorem{proposition}{Proposition}
\Addlcwords{for a is but and with of in as the etc on to if}

\newcommand\github{\includegraphics[height=2ex]{imgs/github.png}}
\newcommand\email{\includegraphics[height=3ex]{imgs/email.png}}

\setbeamertemplate{section in toc}{\inserttocsectionnumber.~\inserttocsection}
\usetheme{Boadilla}
\makeatletter
\setbeamertemplate{footline}{%
    \leavevmode%
    \hbox{%
        \begin{beamercolorbox}[wd=.3\paperwidth,ht=2.25ex,dp=1ex,center]{author in head/foot}%
            \usebeamerfont{author in head/foot}\insertshortauthor\expandafter\beamer@ifempty\expandafter{\beamer@shortinstitute}{}{~~(\insertshortinstitute)}
        \end{beamercolorbox}%
        \begin{beamercolorbox}[wd=.55\paperwidth,ht=2.25ex,dp=1ex,center]{title in head/foot}%
            \usebeamerfont{title in head/foot}\insertshorttitle
        \end{beamercolorbox}%
        \begin{beamercolorbox}[wd=.15\paperwidth,ht=2.25ex,dp=1ex,right]{date in head/foot}%
            \usebeamerfont{date in head/foot}\insertshortdate{}\hspace*{2em}
            \insertframenumber{} / \inserttotalframenumber\hspace*{2ex} 
        \end{beamercolorbox}}%
        \vskip0pt%
    }
\makeatother

\begin{document}

\title{Testing and Estimation in Causal Models}
\subtitle{Addressing Mixed and Missing Data Challenges}
\author{Ankur Ankan}
\institute[]{Radboud University, The Netherlands}
\date{}

\maketitle

\begin{frame}{Causal Inference}
	In many scientific domains, causal questions are of interest.

	\todo[inline]{Add examples here}

	Interested in: 1. Understanding how the variable interact. 2) To estimate effects.
	
	The best way to answer this is using randomized controlled trials. but not always possible.
\end{frame}

\begin{frame}{Directed Acyclic Graphs (DAGs)}

	To answer such questions using only observational data we need to understand the data generating process.

	\todo[inline]{Show an example of a DAG}

	Even though there are many algorithms to learn these graphs, in practice these graphs are constructed manually.
\end{frame}

\begin{frame}{Chapter 2: Testing DAGs}
	Because of being constructed manually important to test these models.

	Chapter 2 gives a tutorial on how to test these.
\end{frame}

\begin{frame}{Chapter 3: CI test for testing and constructing graph}
	The model testing utilizes conditional independence tests.

	In Chapter 3, we propose a novel test for mixed data.
\end{frame}

\begin{frame}{Chapter 4: Estimation}
	Before doing estimation, we need to come up with a way 
\end{frame}

\begin{frame}{Chapter 5: A software tool}
	To make it easy for these methods to use, we have a software package.
\end{frame}

\begin{frame}{Future research}
\end{frame}

\end{document}
