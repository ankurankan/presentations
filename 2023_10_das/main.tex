\documentclass{beamer}

\usepackage[utf8]{inputenc}
\usecolortheme{beaver}
\usepackage{caption}
\usepackage{subcaption}
\usepackage{mathtools}
\usepackage{todonotes}
\usepackage{amsmath}
\usepackage{bm}
\usepackage{listings}
\usepackage{ragged2e}
\usepackage{titlecaps}
\usepackage{fancyvrb}

\def\ci{\perp\!\!\!\!\!\perp}

\newtheorem{proposition}{Proposition}
\Addlcwords{for a is but and with of in as the etc on to if}

\setbeamertemplate{section in toc}{\inserttocsectionnumber.~\inserttocsection}
\usetheme{Boadilla}
\makeatletter
\setbeamertemplate{footline}{%
    \leavevmode%
    \hbox{%
        \begin{beamercolorbox}[wd=.3\paperwidth,ht=2.25ex,dp=1ex,center]{author in head/foot}%
            \usebeamerfont{author in head/foot}\insertshortauthor\expandafter\beamer@ifempty\expandafter{\beamer@shortinstitute}{}{~~(\insertshortinstitute)}
        \end{beamercolorbox}%
        \begin{beamercolorbox}[wd=.55\paperwidth,ht=2.25ex,dp=1ex,center]{title in head/foot}%
            \usebeamerfont{title in head/foot}\insertshorttitle
        \end{beamercolorbox}%
        \begin{beamercolorbox}[wd=.15\paperwidth,ht=2.25ex,dp=1ex,right]{date in head/foot}%
            \usebeamerfont{date in head/foot}\insertshortdate{}\hspace*{2em}
            \insertframenumber{} / \inserttotalframenumber\hspace*{2ex} 
        \end{beamercolorbox}}%
        \vskip0pt%
    }
\makeatother

\begin{document}

\title[]{Residualization Based Conditional Independence Test for Mixed Data}
\author {Ankur Ankan}
\date{}

\maketitle

\begin{frame}{Talk Outline}
	Two parts:
	\begin{itemize}
		\item A Conditional Independence (CI) test for mixed data.
		\item 
	\end{itemize}
	\footnotetext[1]{Reference}
\end{frame}

\begin{frame}{Applications of CI testing in Causal Inference}
	Why do we care about CI testing: Model testing and structure learning.
\end{frame}

\begin{frame}{Residualization based CI testing: Idea}
	\center Partial correlation test
\end{frame}

\begin{frame}{Residualization-based test}
	Has $ 3 $ parts:
	\begin{itemize}
		\item Two estimators: $ \mathbb{E}[X | Z] $ and $ \mathbb{E}[Y | Z] $.
		\item Method to compute residuals.
		\item A test statistic to test for associaton between the residuals.
	\end{itemize}
\end{frame}

\begin{frame}{Part 1: Estimators for mixed data}
	\begin{itemize}
		\item Plenty of options for mixed data.
		\item Linear regression based for using theoretical propoerties.
		\item Non-parameteric models like Random Forest to deal with noise.
	\end{itemize}
\end{frame}

\begin{frame}{Part 2: Residualization Method: Continuous variables}
	Difference between the true and the predicted values.
\end{frame}

\begin{frame}{Residualization: Ordinal variables}

	No simple definition. Use LS-residuals

	Gives valid residuals.
\end{frame}

\begin{frame}{Residualization: Categorical Variables}
	Use dummy encoding.
\end{frame}

\begin{frame}{Residaulization: Summary}
	Results in two residual matrices of continuous variables.
\end{frame}

\begin{frame}{Test Statistics}
	\begin{enumerate}
		Hotelling's $T^2$ test between the residuals.
	\end{enumerate}
\end{frame}

\begin{frame}{Empirical Calibration and data}
\end{frame}

\begin{frame}{Problems: High dimension and low sample size case}
	Looses calibration.
\end{frame}

\begin{frame}{Problems: No effect size measure}
	Mahalanobis distance is used but not interpretable.
\end{frame}

\begin{frame}{High dimension and low sample size problem}
	Why?
	Because estimating covariance matrices in high dimensions is difficult.
	Show examples and possible solutions.
\end{frame}

\begin{frame}{Canoncial Correlations}
	Problem getting coverted into testing association between two 
	sets of random variables.

	Canonical correlations are the most generalized measure.
\end{frame}

\begin{frame}{Pillai's Trace}
	Give the pillai's trace and show plots.
\end{frame}

\begin{frame}{Effect Size}
	Give the effect size for pillai's trace.
\end{frame}

\begin{frame}{Path analysis}
\end{frame}

\begin{frame}{Generalizing path analysis to mixed data}
\end{frame}

\begin{frame}{Doing path analysis using cnaonical correlations}
\end{frame}

\begin{frame}{Simulating mixed data from canoncial correlation parameters}
\end{frame}

\begin{frame}{Conclusions and problems}
\end{frame}

\end{document}
