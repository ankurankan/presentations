\documentclass{beamer}

\usepackage[utf8]{inputenc}
\usecolortheme{beaver}
\usepackage{caption}
\usepackage{subcaption}
\usepackage{mathtools}
\usepackage{todonotes}
\usepackage{amsmath}
\usepackage{bm}
\usepackage{listings}
\usepackage{ragged2e}
\usepackage{titlecaps}
\usepackage{fancyvrb}

\def\ci{\perp\!\!\!\!\!\perp}

\newtheorem{proposition}{Proposition}
\Addlcwords{for a is but and with of in as the etc on to if}

\setbeamertemplate{section in toc}{\inserttocsectionnumber.~\inserttocsection}
\usetheme{Boadilla}
\makeatletter
\setbeamertemplate{footline}{%
    \leavevmode%
    \hbox{%
        \begin{beamercolorbox}[wd=.3\paperwidth,ht=2.25ex,dp=1ex,center]{author in head/foot}%
            \usebeamerfont{author in head/foot}\insertshortauthor\expandafter\beamer@ifempty\expandafter{\beamer@shortinstitute}{}{~~(\insertshortinstitute)}
        \end{beamercolorbox}%
        \begin{beamercolorbox}[wd=.55\paperwidth,ht=2.25ex,dp=1ex,center]{title in head/foot}%
            \usebeamerfont{title in head/foot}\insertshorttitle
        \end{beamercolorbox}%
        \begin{beamercolorbox}[wd=.15\paperwidth,ht=2.25ex,dp=1ex,right]{date in head/foot}%
            \usebeamerfont{date in head/foot}\insertshortdate{}\hspace*{2em}
            \insertframenumber{} / \inserttotalframenumber\hspace*{2ex} 
        \end{beamercolorbox}}%
        \vskip0pt%
    }
\makeatother

\begin{document}

\title[]{Statistical and Causal Robustness for Causal Null Hypothesis Tests}
\date{}

\maketitle

\begin{frame}{Background}
	Given a specified model (i.e., DAG), lots of methods to check for identification.

	Pearl's do-calculus gives a complete solution to both checking for identification and getting an estimator.

	But hard to apply in practice.
	
	% TODO: An example of an identified and non-identified model.

	Backdoor Criterion, Front-door criterion, IV, AIPW. Given the model, estimators for each of them are statistical robust i.e. unbiased, fast convergence rates, etc.
	All of them make different assumptions.
	These assumptions are hard to test in observational data.
\end{frame}

\begin{frame}{Background: Examples}
	\begin{columns}
		\begin{column}{0.33 \textwidth}
			\center
			\begin{figure}
				$ A \rightarrow Y; A \leftarrow C \rightarrow Y $
				\caption{Backdoor criterion: No unobserved confounding.}
			\end{figure}
		\end{column}
		\begin{column}{0.33 \textwidth}
			\center
			\begin{figure}
				$ A \rightarrow M \rightarrow Y; A \leftarrow U \rightarrow Y  $
				\caption{Front-door: All effect mediated through $ A \rightarrow M $ edge.}
			\end{figure}
		\end{column}
		\begin{column}{0.33 \textwidth}
			\center
			\begin{figure}
				$ Z \rightarrow A; A \rightarrow Y; U \leftarrow A \rightarrow Y $
				\caption{Instrumental Variables; No correlation between $ Z $ and $ Y $ except going through $ A $.}
			\end{figure}
		\end{column}
	\end{columns}
\end{frame}

\begin{frame}{Problem Statement}
	\begin{itemize}
		\item It is challenging to come up with a single correct specification of the model.
		\item Instead, use a bunch of different model. And the paper proposes a robustness test on this set of models.
		\item The test check if the causal assumption is true in atleast one of the specified model.
		\item They use something called Evidence Factors to combine the results from testing on the models.
		\item Limited to semi-parameteric estimation methods. For convergence properties of the test??
	\end{itemize}
\end{frame}

\begin{frame}{Example}
	% TODO: Talk about the example they show in the paper.
\end{frame}

\begin{frame}{Evidence Factors}

\end{frame}

\begin{frame}{Proposed Solution}
\end{frame}

\end{document}
