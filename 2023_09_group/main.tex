\documentclass{beamer}

\usepackage[utf8]{inputenc}
\usecolortheme{beaver}
\usepackage{caption}
\usepackage{subcaption}
\usepackage{mathtools}
\usepackage{todonotes}
\usepackage{amsmath}
\usepackage{bm}
\usepackage{listings}
\usepackage{ragged2e}
\usepackage{titlecaps}
\usepackage{fancyvrb}

\def\ci{\perp\!\!\!\!\!\perp}

\newtheorem{proposition}{Proposition}
\Addlcwords{for a is but and with of in as the etc on to if}

\setbeamertemplate{section in toc}{\inserttocsectionnumber.~\inserttocsection}
\usetheme{Boadilla}
\makeatletter
\setbeamertemplate{footline}{%
    \leavevmode%
    \hbox{%
        \begin{beamercolorbox}[wd=.3\paperwidth,ht=2.25ex,dp=1ex,center]{author in head/foot}%
            \usebeamerfont{author in head/foot}\insertshortauthor\expandafter\beamer@ifempty\expandafter{\beamer@shortinstitute}{}{~~(\insertshortinstitute)}
        \end{beamercolorbox}%
        \begin{beamercolorbox}[wd=.55\paperwidth,ht=2.25ex,dp=1ex,center]{title in head/foot}%
            \usebeamerfont{title in head/foot}\insertshorttitle
        \end{beamercolorbox}%
        \begin{beamercolorbox}[wd=.15\paperwidth,ht=2.25ex,dp=1ex,right]{date in head/foot}%
            \usebeamerfont{date in head/foot}\insertshortdate{}\hspace*{2em}
            \insertframenumber{} / \inserttotalframenumber\hspace*{2ex} 
        \end{beamercolorbox}}%
        \vskip0pt%
    }
\makeatother


\begin{document}

\title[]{Conditional Independence Testing via Latent Reprsentation Learning}
\author {}
\date{}

\maketitle
\begin{frame}{Conditional Indpendence}
	\begin{block}{Independence}
		$$ X \ci Y \implies P(X, Y) = P(X) * P(Y) $$	
	\end{block}
	\begin{block}{Conditional Independence}
		$$ X \ci Y | Z  \implies P(X, Y | Z) = P(X | Z ) *  P(Y | Z) $$
	\end{block}
\end{frame}
\begin{frame}{Residualization method for CI}
	\begin{itemize}
		\item Intuition: If we somehow take out all the effect of $ Z $
			from both $ X $ and $ Y $, and if the remaining $ X $
			and $ Y $ are independent, it implies that they are
			conditionally independent.
		\item This is one class of CI test.
		\item Simplest example is the partial correlation test. Trains
			two linear regression models $ E_X = X \sim Z $ and $
			E_Y = Y \sim Z $ and takes the residuals from these two models 
			$ R_X = X - \hat{X} $ and $ R_Y = Y - \hat{Y} $ and does a
			correlation test on $ R_X $ and $ R_Y $.
		\item Make different variations to this approach:
			1) Generalized Covariance Measure (GCM) by Shah and Peters
			2) Based on Hotelling's T^2 test by us.
	\end{itemize}
\end{frame}

\end{document}
