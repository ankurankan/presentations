\documentclass{beamer}

\usepackage[utf8]{inputenc}
\usecolortheme{beaver}
\usepackage{caption}
\usepackage{subcaption}
\usepackage{mathtools}
\usepackage{todonotes}
\usepackage{amsmath}
\usepackage{bm}
\usepackage{listings}
\usepackage{ragged2e}
\usepackage{titlecaps}
\usepackage{fancyvrb}

\def\ci{\perp\!\!\!\!\!\perp}

\newtheorem{proposition}{Proposition}
\Addlcwords{for a is but and with of in as the etc on to if}

\setbeamertemplate{section in toc}{\inserttocsectionnumber.~\inserttocsection}
\usetheme{Boadilla}
\makeatletter
\setbeamertemplate{footline}{%
    \leavevmode%
    \hbox{%
        \begin{beamercolorbox}[wd=.3\paperwidth,ht=2.25ex,dp=1ex,center]{author in head/foot}%
            \usebeamerfont{author in head/foot}\insertshortauthor\expandafter\beamer@ifempty\expandafter{\beamer@shortinstitute}{}{~~(\insertshortinstitute)}
        \end{beamercolorbox}%
        \begin{beamercolorbox}[wd=.55\paperwidth,ht=2.25ex,dp=1ex,center]{title in head/foot}%
            \usebeamerfont{title in head/foot}\insertshorttitle
        \end{beamercolorbox}%
        \begin{beamercolorbox}[wd=.15\paperwidth,ht=2.25ex,dp=1ex,right]{date in head/foot}%
            \usebeamerfont{date in head/foot}\insertshortdate{}\hspace*{2em}
            \insertframenumber{} / \inserttotalframenumber\hspace*{2ex} 
        \end{beamercolorbox}}%
        \vskip0pt%
    }
\makeatother

\begin{document}

\title[]{Enhancing and Promoting Data Simulation Capabilities of pgmpy}
\date{}

\maketitle

\begin{frame}{Directed Acyclic Graphs}

%TODO: Insert a figure of a real DAG.

Bayesian Networks (BNs) or Directed Acyclic Graphs (DAGs) represent relationship between variables using a graphical structure. Used widely in a lot of fields ranging from %TODO: Finish this

Userbase can be very broadly categorized into: 
\begin{enumerate}
	\item Predictive Modelling: Used as a way to efficiently represent
		joint distributions and perform operations on it. The graphical
		structure adds interpretability.
	\item Causal Inference: With some added assumptions, used to answer
		causal questions such as: what would be the effect of changing etc.
		%TODO: add causal questions based on the example.
\end{enumerate}

\end{frame}

\begin{frame}{Use of simulations in these graphical models}
%TODO: Make a table here with Predictive modelling and causal inference columns. And put each use case in either or middle for both.

Common between both: Structure Learning/Causal Discovery, Education.
Predictive modelling: Approximate inference.
Causal Inference: People are interested in testing how well different estimation methods perform.
\end{frame}

\begin{frame}{Simulation features in pgmpy}
	Allows simulation from these BNs under various conditions such as (uncertain) evidence, (uncertain) intervention, time-series data, partial simulations for incorporating other simulations.

	Wide range of features but is limited to discrete variable.

	Extending this to continuous variables would have a huge impact as it would allow for 
	a lot of different simulations. No other Python package that offers so much functionality.
\end{frame}

\begin{frame}{Fellowship Plan: Phase 1}
	Extend the simulation capabilties to continuous variables.

	%TODO: Show a figure to show the following message:

	The parameterization is represented in tabular CPDs which is essentially a function of the variable given it's parents. => Extend pgmpy such that it can accept arbitrary functions.
\end{frame}

\begin{frame}{Fellowship Plan: Phase 2}
	Promote these data simulation capabilities.

	1) Tutorial for 
\begin{frame}

\begin{frame}{Why am I suitable to perform this}
	
\end{frame}

\end{document}
