\documentclass{beamer}

\usepackage[utf8]{inputenc}
\usecolortheme{beaver}
\usepackage{caption}
\usepackage{subcaption}
\usepackage{mathtools}
\usepackage{todonotes}
\usepackage{amsmath}
\usepackage{bm}
\usepackage{listings}
\usepackage{ragged2e}
\usepackage{fancyvrb}

\def\ci{\perp\!\!\!\!\!\perp}

\newtheorem{proposition}{Proposition}

\setbeamertemplate{section in toc}{\inserttocsectionnumber.~\inserttocsection}
\usetheme{Boadilla}
\makeatletter
\setbeamertemplate{footline}{%
    \leavevmode%
    \hbox{%
        \begin{beamercolorbox}[wd=.3\paperwidth,ht=2.25ex,dp=1ex,center]{author in head/foot}%
            \usebeamerfont{author in head/foot}\insertshortauthor\expandafter\beamer@ifempty\expandafter{\beamer@shortinstitute}{}{~~(\insertshortinstitute)}
        \end{beamercolorbox}%
        \begin{beamercolorbox}[wd=.55\paperwidth,ht=2.25ex,dp=1ex,center]{title in head/foot}%
            \usebeamerfont{title in head/foot}\insertshorttitle
        \end{beamercolorbox}%
        \begin{beamercolorbox}[wd=.15\paperwidth,ht=2.25ex,dp=1ex,right]{date in head/foot}%
            \usebeamerfont{date in head/foot}\insertshortdate{}\hspace*{2em}
            \insertframenumber{} / \inserttotalframenumber\hspace*{2ex} 
        \end{beamercolorbox}}%
        \vskip0pt%
    }
\makeatother

\begin{document}

\title[Combining Graphical and Algebraic Approaches for Parameter Identification in Latent Variable SEM]{Combining Graphical and Algebraic Approaches for Parameter Identification in Latent Variable Structural Equation Models}
\author {Ankur Ankan \and Johannes Textor}
\institute[]{Radboud University, Netherlands}
\date{}
\maketitle

\section{Overview}
\begin{frame}
	\frametitle{The identfication problem}
\end{frame}

\begin{frame}
	\frametitle{Instrumental Variable(IV) Estimation}
\end{frame}

\begin{frame}
	\frametitle{Overview}
	Graphical Models IV criteria $\downarrow $ \\
	IV estimation in SEM $ \rightarrow $ Graphical L2O transformation $ \rightarrow $ Graphical identification for SEM $ \rightarrow $ Can apply any graphical criteria to SEMs; Equivalence proof between Instrumental Sets and MIIV approach; Relaxing normality condition for graphical criteria.
\end{frame}

\begin{frame}
	\frametitle{Graphical Approach}
\end{frame}

\begin{frame}
	\frametitle{Why can't we directly apply graphical criteria to SEMs}
	\begin{itemize}
		\item Criterion have been developed assuming that the parameters 
			begin identified are between observed variables.
		\item Show an example of a SEM where we are looking to identify 
			the parameter on latent variables.
	\end{itemize}
\end{frame}

\begin{frame}
	\frametitle{MIIV Approach}
	\begin{itemize}
		\item The MIIV approach deals with these cases by doing a
			Latent-to-Observed (L2O) transformation.
	\end{itemize}
\end{frame}

\begin{frame}
	\frametitle{The proposed graphical transformation}
\end{frame}

\begin{frame}
	\frametitle{Examples}
\end{frame}

\end{document}
