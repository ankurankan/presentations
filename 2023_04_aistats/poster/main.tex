\documentclass{beamer}
\usepackage[orientation=portrait,width=84.1cm,height=59.4cm,scale=1.4,debug]{beamerposter}
\mode<presentation>{\usetheme{ZH}}
\usepackage{chemformula}
\usepackage[utf8]{inputenc}
\usepackage[english]{babel} % required for rendering German special characters
\usepackage{hyperref} %enable hyperlink for urls
\usepackage{ragged2e}
\usepackage[font=small, justification=justified]{caption}
\usepackage{array,booktabs,tabularx}
\usepackage{bm}
\usepackage{subcaption}
\usepackage{titlecaps}
\usepackage{natbib}

\Addlcwords{for a is but and with of in as the etc on to if}
\def\ci{\perp\!\!\!\!\!\perp}
\tikzstyle{latent} = [ draw, circle, inner sep = 5pt, minimum size = 1.5cm, line width=1mm]
\tikzstyle{observed} = [ draw, rectangle, inner sep = 5pt, minimum size = 1.5cm, line width=1mm ]

\newcolumntype{Z}{>{\centering\arraybackslash}X} % centered tabularx columns
\title{\huge Combining Graphical and Algebraic Approaches for Parameter Identification in Latent Variable Structural Equation Models}
\author{Ankur Ankan \inst{1} \and Inge Wortel \inst{1} \and Kenneth A. Bollen \inst{2} \and Johannes Textor \inst{1}}
\institute[]{\inst{1} Institute for Computing and Information Sciences, Radboud University \\ \inst{2} Department of Psychology and Neuroscience, University of North Carolina at Chappel Hill}
\date{\today}

% edit this depending on how tall your header is. We should make this scaling automatic :-/
\newlength{\columnheight}
\setlength{\columnheight}{106cm}

\begin{document}
\begin{frame}
\begin{columns}
	\begin{column}{.33\textwidth}
		\begin{beamercolorbox}[center]{postercolumn}
			\begin{minipage}{.98\textwidth}  % tweaks the width, makes a new \textwidth
				\parbox[t][\columnheight]{\textwidth}{ % must be some better way to set the the height, width and textwidth simultaneously
	\begin{myblock}{\titlecap{Introduction}}
		\justifying \textbf{Parameter Identification:} Is the causal parameter(s) estimable from observational data?
			\begin{figure}
				\centering
				\begin{subfigure}{0.5 \textwidth}
					\centering
					\begin{tikzpicture}[scale=1.5]
						\tikzstyle{every node}=[align=center, inner sep=1pt]
						\node[latent] (u) at (1.5, 1.5) {$ U $};
						\node[observed] (x) at (0, 0) {$ X $};
						\node[observed] (y) at (3, 0) {$ Y $};
						\node[text=red] (text) at (3.5, 2.5) {Latent};
						\draw[->, line width=1mm] (u) -- (x);
						\draw[->, line width=1mm] (u) -- (y);
						\draw[->, line width=1mm] (x) -- (y) node[midway, below]{$ \beta $};
						\draw[->, line width=1mm, red] (text) to [out=220, in=40] (u);
					\end{tikzpicture}
					\caption*{\textbf{$ \beta $ is not identified}}
				\end{subfigure}%
				\begin{subfigure}{0.5 \textwidth}
					\centering
					\begin{tikzpicture}[scale=1.5]
						\tikzstyle{every node}=[align=center, inner sep=1pt]
						\node[observed] (u) at (1.5, 1.5) {$ U $};
						\node[observed] (x) at (0, 0) {$ X $};
						\node[observed] (y) at (3, 0) {$ Y $};
						\node[text=red] (text) at (3.5, 2.5) {Observed};
						\draw[->, line width=1mm] (u) -- (x);
						\draw[->, line width=1mm] (u) -- (y);
						\draw[->, line width=1mm] (x) -- (y) node[midway, below]{$ \beta $};
						\draw[->, line width=1mm, red] (text) to [out=220, in=40] (u);
					\end{tikzpicture}
					\caption*{\textbf{$ \beta $ is identified}}
				\end{subfigure}
			\end{figure}
			\justifying \textbf{Instrumental Variables (IV):} Can be used to estimate causal effect in the presence of confounding.
				\begin{figure}
					\centering
					\begin{subfigure}{0.5\linewidth}
						\centering
						\begin{tikzpicture}[scale=2]
							\tikzstyle{every node}=[align=center, inner sep=1pt]
							\node[observed] (z) at (0, 0) {$ Z $};
							\node[observed] (x) at (2, 0) {$ X $};
							\node[observed] (y) at (4, 0) {$ Y $};
							\node[latent] (u) at (3, 1) {$ U $};
							\draw[->, line width=1mm] (z) -- (x) node[midway, below] {$ \alpha $};
							\draw[->, line width=1mm] (x) -- (y) node[midway, below] {$ \beta $};
							\draw[->, line width=1mm] (u) -- (x);
							\draw[->, line width=1mm] (u) -- (y);
						\end{tikzpicture}
						\caption*{}
					\end{subfigure}%
					\begin{subfigure}{0.5\linewidth}
						\centering
						$$ \beta = \frac{Cov(Y, Z)}{Cov(Z, X)} = \frac{\alpha \beta}{\alpha}$$
						\caption*{}
					\end{subfigure}
				\end{figure}
			\vspace{-0.9em}
			$ Z $ qualifies as an IV for $ X \to Y $ if: 1)
			All open paths from $ Z $ to $ Y $ are mediated through
			$ X $. b) $ Z $ is not independent of $ X $.
		\vspace{0.9em}

		\begin{itemize}
			\item \justifying Many graphical IV criteria have been
				proposed for Directed Acyclic Graphs (DAGs).
				E.g., Instrumental Sets, Instrumental Cutsets,
				Auxiliary Variables, etc.
			\item \justifying The graphical criteria assume that
				the parameter(s) of interest are between
				observed variables, hence are not directly
				applicable to latent variable Structural Equation Models(SEMs).
			\item \justifying A \textbf{Model Implied Instrumental
				Variable (MIIV)} approach has been proposed for
				SEMs.
			\item \justifying MIIV approach deals with latent
				variables by using \textbf{Scaling
				Indicators} to do \textbf{Latent-to-Observed (L2O)}
				transformations.
		\end{itemize}
	\end{myblock}\vfill
	\begin{myblock}{\titlecap{Background}}
		\textbf{SEM:} A system of linear equations.
			\begin{equation*}
				\begin{split}
					\bm{X} &= \bm{BX} + \bm{E} \\
					x_i &= \epsilon_i + \sum_{x_j \in Co(x_i)} \beta_{x_j \to x_i} x_j 
				\end{split}
			\end{equation*}

		\vspace{0.3em}
		\justifying \textbf{Scaling Indicator:} Fixes the scale of each latent variable to an arbitrary value (e.g., 1).
			$$ x_j = x_j^s - \epsilon_{x_j^s} $$

		\vspace{0.3em}
		\justifying \textbf{\titlecap{Algebraic L2O transformation:}} Replaces all latent variables with their scaling indicators.
 			$$ x_i = \epsilon_i + \sum_{x_j \in Co_l(x_i)} \beta_{x_j \rightarrow x_i} (x_j^s - \epsilon_{x_j^s}) + \sum_{x_k \in Co_o(x_i)} \beta_{x_k \rightarrow x_i} x_k $$

		\vspace{0.3em}
		\justifying \textbf{\titlecap{Partial Algebraic L2O transformation:}} Replaces latents only for parameter(s) of interest.
			\begin{equation*}
			\begin{split}
				x_i = \epsilon_i + & \sum_{x_j \in Co_l^i(x_i)} \beta_{x_j \rightarrow x_i} (x_j^s - \epsilon_{x_j^s}) +  \\
						   & \sum_{x_k \in Co_l(x_i) \setminus Co_l^i(x_i)} \beta_{x_k \rightarrow x_i} x_k + \sum_{x_l \in Co_o(x_i)} \beta_{x_l \rightarrow x_i} x_l
			\end{split}
			\end{equation*}
	\end{myblock} \vfill
	\begin{myblock}{\titlecap{Proposed Graphical L2O Transformation}}
		\begin{enumerate}
			\item \textbf{Latent-to-Observed arrow:}
				\begin{figure}
					\centering
					\includegraphics[page=4, scale=3]{../slides/figures-inge.pdf}
					\label{fig:l2o_lat_obs}
				\end{figure}
				\begin{equation*}
					\begin{split}
						y_3 &= \beta l_1 + \beta_{y_5 \to y_3} y_5 + \epsilon_3 \\
						l_1 &= y_2 - \epsilon_{2} + \beta_{y_1 \to y_2} y_1 \\
						y_3 &= \beta y_2 - \beta \beta_{y_1 \to y_2} y_1 + \beta_{y_5 \to y_3} y_5 - \beta \epsilon_2 + \epsilon_3\\
					\end{split}
				\end{equation*}
		\end{enumerate}
	\end{myblock}\vfill
		}\end{minipage}\end{beamercolorbox}
	\end{column}

%%%%%%%%%%%%%%%%%%%%%%%%%%%%%%%%%%%%%%%%%%% Second Column %%%%%%%%%%%%%%%%%%%%%%%%%%%%%%%%%%%%%%%%%%%%%%%%%%%%%%%

	\begin{column}{.33\textwidth}
 		\begin{beamercolorbox}[center]{postercolumn}
 			\begin{minipage}{.98\textwidth} % tweaks the width, makes a new \textwidth
 				\parbox[t][\columnheight]{\textwidth}{ % must be some better way to set the the height, width and textwidth simultaneously
					\begin{myblock}{\titlecap{Proposed Graphical L2O Transformation (Cont.)}}
						\begin{enumerate}
							\setcounter{enumi}{1}
							\item \textbf{Observed-to-Latent arrow:}
								\begin{figure}
									\centering
									\includegraphics[page=5, scale=3]{../slides/figures-inge.pdf}
									\label{fig:l2o_obs_lat}
								\end{figure}
								\begin{equation*}
									\begin{split}
										y_4 &= l_1 + \beta_{y_2 \to y_4} y_2 + \beta_{y_3 \to y_4} + \epsilon_4 \\
										l_1 &= \beta y_1 + \zeta_1 \\
										y_4 &= \beta y_1 + \beta_{y_3 \to y_4} y_3 + \beta_{y_2 \to y_4} y_2 + \zeta_1 + \epsilon_4 \\
									\end{split}
								\end{equation*}
							\vspace{0.6em}	
							\item \textbf{Latent-to-Latent arrow:}
								\begin{figure}
									\centering
									\includegraphics[page=6, scale=3]{../slides/figures-inge.pdf}
									\label{fig:l2o_lat_lat}
								\end{figure}
								\begin{equation*}
									\begin{split}
										l_2 &= \beta l_1 + \zeta_2 \\
										y_2 &= l_2 + \epsilon_2; y_1 = l_1 + \epsilon_1 \\
										y_2 &= \beta y_1 - \beta \epsilon_1 + \zeta_2 + \epsilon_2
									\end{split}
								\end{equation*}
						\end{enumerate}
					\end{myblock}\vfill
					\begin{myblock}{\titlecap{Examples of Graphical Identification in SEMs}}
						\begin{figure}
							\begin{subfigure}{0.5 \linewidth}
								\centering
								\includegraphics[page=7, scale=3]{../slides/figures-inge.pdf}
							\end{subfigure}%
							\begin{subfigure}{0.5 \linewidth}
								\centering
								\includegraphics[page=8, scale=3]{../slides/figures-inge.pdf}
							\end{subfigure}
							\caption*{\textbf{\textcolor{black}{Full Equation Identification:}} \textbf{(Left)} Example model; \textbf{(Right)} L2O 
										transformed model for the equation $ y_3 = l_1 + l_2 $. Both the parameters $ \lambda_{13} $ and 
										$ \lambda_{23} $ are identified using both MIIV and Instrumental Sets.}
						\end{figure}
						\vspace{0.6em}
						\begin{figure}
							\begin{subfigure}{0.33 \linewidth}
								\centering
								\includegraphics[page=11, scale=3]{../slides/figures-inge.pdf}
							\end{subfigure}%
							\begin{subfigure}{0.33 \linewidth}
								\centering
								\includegraphics[page=12, scale=3]{../slides/figures-inge.pdf}
							\end{subfigure}%
							\begin{subfigure}{0.33 \linewidth}
								\centering
								\includegraphics[page=13, scale=3]{../slides/figures-inge.pdf}
							\end{subfigure}
							\caption*{\textbf{\textcolor{black}{Partial Equation Identification:}} \textbf{(Left)} Example model; 
										\textbf{(Center)} L2O transformed model for the equation $ y_3 = l_1 + l_2 $. The parameters 
										$ \lambda_{23} $ and $ \lambda_{13} $ are not identified together; \textbf{(Right)} L2O 
										transformed model for the partial equation, $ y_3 = l_2 $. $ \lambda_{23} $ is identified in this case.
										}
						\end{figure}
						\vspace{0.6em}
						\begin{figure}
							\begin{subfigure}{0.5 \linewidth}
								\centering
								\includegraphics[page=14, scale=3]{../slides/figures-inge.pdf}
							\end{subfigure}%
							\begin{subfigure}{0.5 \linewidth}
								\centering
								\includegraphics[page=15, scale=3]{../slides/figures-inge.pdf}
							\end{subfigure}
							\caption*{\textbf{\textcolor{black}{Identification using Conditional IVs:}} \textbf{(Left)} Original model;
									\textbf{(Right)} L2O transformed model for the partial equation, $ y_3 = l_2 $. $ \lambda_{23} $ is not 
									identified through simple IVs but $y_5$ can be used as a conditional IV when conditioned on $ y_6 $.}						\end{figure}
					\end{myblock}\vfill
 				}
			\end{minipage}
		\end{beamercolorbox}
	\end{column}


%%%%%%%%%%%%%%%%%%%%%%%%%% Third column %%%%%%%%%%%%%%%%%%%%%%%%%%%%%%%%%%%%%%%%%%%%%%%%%%%%%
	\begin{column}{0.33\textwidth}
		\begin{beamercolorbox}[center]{postercolumn}
			\begin{minipage}{.98\textwidth} % tweaks the width, makes a new \textwidth
				\parbox[t][\columnheight]{\textwidth}{ % must be some better way to set the the height, width and textwidth simultaneously
				\begin{myblock}{\titlecap{Examples of Graphical Identification in SEMs}}
					\begin{figure}
						\begin{subfigure}{0.33 \linewidth}
							\centering
							\includegraphics[page=16, scale=3]{../slides/figures-inge.pdf}
						\end{subfigure}%
						\begin{subfigure}{0.33 \linewidth}
							\centering
							\includegraphics[page=17, scale=3]{../slides/figures-inge.pdf}
						\end{subfigure}%
						\begin{subfigure}{0.33 \linewidth}
							\centering
							\includegraphics[page=18, scale=3]{../slides/figures-inge.pdf}
						\end{subfigure}
						\caption*{\textbf{\textcolor{black}{Inestimable Parameters in Identified Equation:}} \textbf{(Left)} Example model 
						from \citet{griliches1977estimating}; \textbf{(Center)} Slightly modified version; \textbf{(Right)} L2O transformed
						model for the equation, $ y_4 = l_1 + y_2 + y_3 $. Even though the equation is identified, $ \lambda_{13} $ and 
						$ \lambda_{34} $ cannot be individually estimated.}
					\end{figure}
				\end{myblock}
	\begin{myblock}{\titlecap{MIIV is equivalent to Instrumental sets}}
		\textbf{Instrumental Sets \citep{BritoP02}:}
		Given an ADMG $\cal
			G$, a variable $y$, and a subset $X$ of the parents of $y$, 
			a set of variables
			$I$ fulfills the 
			\emph{instrumental set condition}
			if for {some} permutation $ i_1 \ldots i_k $ of
			$ I $ and {some} permutation
			$ x_1 \ldots x_k $ of $ X $ we have: 
			\begin{enumerate}
				\item There are no treks from $I$ to $y$ in the graph ${\cal
					G}_{\overline{X}}$ obtained by removing all arrows 
					between $X$ and $y$. 
				\item For each $j$, $1 \leq j \leq k$, there is a trek $\pi_j$ from
					$I_j$ to $X_j$ such that for all $i < j$: (1) $I_i$ does not
					occur on any trek $\pi_j$; and (2) all intersections between
					$\pi_i$ and $\pi_j$ are on the left side of $\pi_i$ and the
					right side of $\pi_j$.
			\end{enumerate}
		\vspace{1em}

		\textbf{Permutation-free Instrumental Sets:}
			Given an ADMG $\cal G$, a variable $y$ and a subset $X$ of the parents
			of $y$, a set of variables $I$ fulfills the \emph{permutation-free
			instrumental set condition} if: (1) There are no treks from $I$ to $y$
			in the graph ${\cal G}_{\overline{X}}$ obtained by removing all arrows
			leaving $X$, and (2) All $t$-separators $(L,R)$ of $I$ and $X$ have
			size $\geq k$.
		\vspace{1em}

		\textbf{Algebraic Instrumental Sets \citep{bollen2012instrumental}:}
			Given a regression equation $y = B \cdot X + \epsilon$, where $X$ possibly
			correlates with $\epsilon$, a set of variables
			$I$ fulfills the \emph{algebraic instrumental set condition} if: (1) $I \ci \epsilon$,
			(2) $\textrm{rk}(\bm{\Sigma}[I,X]) = |X|$, and (3) $\textrm{rk}(\bm{\Sigma}[I]) = |I|$
		
	\end{myblock}\vfill
	\begin{myblock}{\titlecap{Conclusion}}
		\begin{itemize}
			\item \justifying We extend the L2O transformation to do partial equation identification.
			\item \justifying We propose a graphical L20 transformation allowing us to apply graphical identification criteria to latent variable SEMs.
			\item \justifying Application of graphical criteria leads to identification of more parameter compared to MIIV approach.
			\item \justifying Further, we show that the Instrumental set criterion is equivalent to the MIIV approach.
			\item \justifying This equivalence proof also allows us to relax the normality requirement in graphical criterion.
			\item \justifying We bridge the two fields.
		\end{itemize}
	\end{myblock}\vfill
	\begin{myblock}{References}
		\footnotesize
		\bibliographystyle{apalike}
		\bibliography{./bib}
	\end{myblock}\vfill
		}\end{minipage}\end{beamercolorbox}
	\end{column}
\end{columns}
\end{frame}
\end{document}
