\documentclass{beamer}

\usepackage[utf8]{inputenc}
\usecolortheme{beaver}
\usepackage{caption}
\usepackage{amsmath}
\usepackage{amsfonts}
\usepackage{amssymb}
\usepackage{subcaption}
\usepackage{mathtools}
\usepackage[style=verbose, backend=biber]{biblatex}

\begin{document}

\title{Discussion: Partially Adaptive Regularized Multiple Regression Analysis for Estimating Linear Causal Effects}

\begin{frame}
	\frametitle{Recap}
	\begin{itemize}
		\item Deals with the case when multi-colinearirty is present in the regression equation.
		\item Many ways to deal with it in the non-causal case but they don't work well for causal estimation task.
		\item Normal methods may remove some of the covariates from the regression equation depending on it's parameter value,
			but for causal estimation it is important to keep them in the equation.
		\item Collapsibility condition can be derived to show that the proposed method is consistent with OLS.
	\end{itemize}
\end{frame}

\begin{frame}
	\frametitle{Positive points}
	\begin{itemize}
		\item 
	\end{itemize}
\end{frame}

\begin{frame}
	\frametitle{Negative Points}
	\begin{itemize}
		\item More intuition for why it's working.
		\item Further compare other methods for selecting variables.
	\end{itemize}
\end{frame}

\begin{frame}
	\begin{itemize}
		\item Does selection of the variables in the confounding and non-confounding set matter?
		\item Properties of the optimization problem?
		\item Parameter tuning of the regularized regression analysis?
		\item Applying it to single door criterion?
		\item What other things can be estimated used a regression based approach and whether this would be applicable there?
		\item Extending it to other statistical models like GLM, Generalized estimating equations, and proportional hazards model.
\end{frame}

\end{document}
