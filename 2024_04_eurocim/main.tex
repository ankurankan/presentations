\documentclass{beamer}
\usepackage[orientation=portrait,width=30in,height=40in,scale=1.35,debug]{beamerposter}
\mode<presentation>{\usetheme{ZH}}
\usepackage{chemformula}
\usepackage[utf8]{inputenc}
\usepackage[english]{babel} % required for rendering German special characters
\usepackage{hyperref} %enable hyperlink for urls
\usepackage{ragged2e}
\usepackage[font=small, justification=justified]{caption}
\usepackage{array,booktabs,tabularx}
\usepackage{bm}
\usepackage{natbib}
\bibliographystyle{abbrvnat}
\usepackage{subcaption}

\def\ci{\perp\!\!\!\!\!\perp}

\newcolumntype{Z}{>{\centering\arraybackslash}X} % centered tabularx columns
\title{\huge A Generalized Parameterization for \\ Mixed Data Linear Structural Equation Models}
\author{Ankur Ankan, Johannes Textor}
\institute[RU]{Institute for Computing and Information Sciences \\ Radboud University, Netherlands}
\date{\today}

% edit this depending on how tall your header is. We should make this scaling automatic :-/
\newlength{\columnheight}
\setlength{\columnheight}{106cm}

\begin{document}
\begin{frame}
\begin{columns}
	\begin{column}{.33\textwidth}
		\begin{beamercolorbox}[center]{postercolumn}
			\begin{minipage}{.98\textwidth}  % tweaks the width, makes a new \textwidth
				\parbox[t][\columnheight]{\textwidth}{ % must be some better way to set the the height, width and textwidth simultaneously
	\begin{myblock}{Introduction}
		\begin{figure}
			\includegraphics[page=1, scale=4]{figures.pdf}
			\caption{Example of an SEM. The variables $ \{ X_1, X_2, X_3, X_4 \} $ are continuous with path coefficients denoted by $ \beta_{ij} $.}
			\label{fig:example_sem}
		\end{figure}

		\vspace{1em}
		
		Continuous variable linear Structural Equation Models (SEMs)
		are parameterized using \textbf{path coefficients}, defined as
		the coefficients of a multiple regression model of each
		variable on its parents after scaling all variables to unit
		variance. Path coefficients are \textbf{interpretable}:
		\begin{itemize}
			\item One unit change in $ X_2 $ while keeping $ X_4 $ fixed implies $ \beta_{23} $ change in $ X_3 $. 
			\item Wright's path tracing rules \citep{Wright1934}
				can be applied. For example, the indirect
				effect of $ X_2 $ on $ X_3 $ is $ \beta_{24}
				\beta_{34} $, and the total effect of $ X_2 $ on
				$ X_3 $ is $ \beta_{24} \beta_{34} + \beta_{23}
				$.
		\end{itemize}

		\begin{figure}
			\includegraphics[page=4, scale=4]{figures.pdf}
			\caption{Example of a hypothesized mixed variable SEM
				based on the adult income dataset
				\citep{kohavi1996}. The variables Age and Hours
				Per Week are considered continuous, Education
				and Income are considered ordinal, and Sex and
				Occupation are considered categorical.}
			\label{fig:example_adult}
		\end{figure}

		\vspace{0.5em}

		However, such parameterization is \textbf{not available} for
		mixed data models (Fig.~\ref{fig:example_adult}). If regression
		based methods are used on mixed variables, we get a matrix of
		regression coefficients, which is not easily interpretable.

		\vspace{1em}

		\textbf{Research Question 1:} \textcolor{red}{Can we generalize path coefficients to mixed variable SEMs?}
	\end{myblock}\vfill
	\begin{myblock}{Path Coefficients and Effect Sizes}
		The parameter $ \beta_{23} $ in Fig.~\ref{fig:example_sem} can be estimated using the regression equation:
		
		\begin{equation*}
			X_3 \sim X_2 + X_4
		\end{equation*}
	
		The Frisch-Waugh-Lovell theorem \citep{Frisch1933} provides an alternative way to 
		estimate $ \beta_{23} $ in terms of the residuals of $ X_2 $ and $ X_3 $ while using the remaining covariates as the predictors.

		\begin{equation*}
			\begin{split}
				R_{X_2} &= X_2 - \mathbb{E}[X_2 | X_4] \\
				R_{X_3} &= X_3 - \mathbb{E}[X_3 | X_4] \\
				\hat{\beta}_{23} &: R_{X_3} \sim R_{X_2} \\
			\end{split}
		\end{equation*}

		This can be further reformulated in terms of the correlation coefficient, an effect size measure, between $ R_{X_2} $ and $ R_{X_3} $ as:

		\begin{equation*}
			\hat{\beta}_{23} = cor(R_{X_3}, R_{X_2}) \sqrt{\frac{var(R_{X_3})}{var(R_{X_2})}}
		\end{equation*}

		
		Hence, the application of FWL theorem allowed us to transform
		the estimation of path coefficient in terms of estimation of
		\textbf{effect size} between the residuals of variables of
		interest.
		
	\end{myblock}\vfill
		}\end{minipage}\end{beamercolorbox}
	\end{column}

%%%%%%%%%%%%%%%%%%%%%%%%%%%%%%%%%%%%%%%%%%% Second Column %%%%%%%%%%%%%%%%%%%%%%%%%%%%%%%%%%%%%%%%%%%%%%%%%%%%%%%

	\begin{column}{.33\textwidth}
		\begin{beamercolorbox}[center]{postercolumn}
			\begin{minipage}{.98\textwidth} % tweaks the width, makes a new \textwidth
				\parbox[t][\columnheight]{\textwidth}{ % must be some better way to set the the height, width and textwidth simultaneously
	\begin{myblock}{Effect Size Measure for Mixed Data}
		To extend this parameterization to mixed data, we need two components:

		\begin{enumerate}
			\item \textbf{Residuals for mixed data:} We use the mixed data residualization approach from \citet{Ankan2023}.
			\item \textbf{Effect size measure:} The residualization approach gives matrix of residuals. In multivariate statistics, canonical 
				correlations give a generalized measure of association between two sets of variables.
		\end{enumerate}

		\begin{figure}
			\includegraphics[page=3, scale=4]{figures.pdf}
			\caption{The example from Fig.~\ref{fig:example_adult} parameterized using the proposed canonical correlation based effect size.}
		\end{figure}
	\end{myblock}\vfill
		}\end{minipage}\end{beamercolorbox}
	\end{column}


%%%%%%%%%%%%%%%%%%%%%%%%%% Third column %%%%%%%%%%%%%%%%%%%%%%%%%%%%%%%%%%%%%%%%%%%%%%%%%%%%%
	\begin{column}{0.33\textwidth}
		\begin{beamercolorbox}[center]{postercolumn}
			\begin{minipage}{.98\textwidth} % tweaks the width, makes a new \textwidth
				\parbox[t][\columnheight]{\textwidth}{ % must be some better way to set the the height, width and textwidth simultaneously
	\begin{myblock}{Conditional Independence Testing}
		\textbf{Research Question 2:} \textcolor{red}{Can canonical correlation be used for Conditional Independence(CI) testing?}
		\vspace{1em}

		Residualization based CI tests are based on association between the residuals of the variables of interest. As canonical correlation provides 
		a generalized way to assess that, we can utilize it for CI testing. Various tests of association 
		such tests are available, including Wilks' Lambda, Hotelling-Lawley tracle, and Pillai-Bartlett trace \citep{Pillai1955}. Given two residual 
		matrices $ R_x $ and $ R_y $, the PB test statistic is defined as:

		\begin{equation*}
			V(R_x, R_y) = \sum_{\rho \in \bm{\rho}(R_x, R_y)} \rho^2
		\end{equation*}

		\begin{figure}
			\begin{subfigure}{\textwidth}
				\centering
				\includegraphics[scale=3]{calibration.pdf}
				\caption{}
			\end{subfigure}
			\begin{subfigure}{\textwidth}
				\centering
				\includegraphics[scale=3]{accuracy.pdf}
				\caption{}
			\end{subfigure}
			\caption{(a) Calibration and (b) Accuracy comparison between the standard
				Hotelling’s T2 test \citep{Ankan2023}, Adaptable Regularized Hotelling’s T2 test (ARHT), Pillai-Bartlett (PB)
				Trace, and chi-squared test for CI testing in a low sample size and high cardinality
				scenario. Data has been simulated using a common cause model.
			\label{fig:ci_test}
		\end{figure}

	\end{myblock}\vfill
	\begin{myblock}{Conclusion and Future Work}
		\begin{itemize}
			\item Using the connection between path coefficients and effect size between residuals, we propose an mixed variable SEM parameterization based
				on canonical correlations.
			\item We also show that canonical correlation based tests can be used for CI testing, and it performs especially well in high-dimensional scenarios.
			\item The interpretation of these coefficients is still unclear, and needs further investigation.
			\item 

				More thorough analysis needs to be done on the properties of the CI test and theoretical guarantees need to be given.
		\end{itemize}
	\end{myblock}\vfill
	\begin{myblock}{References}
		\footnotesize
		\bibliography{./bib}
	\end{myblock}\vfill
		}\end{minipage}\end{beamercolorbox}
	\end{column}
\end{columns}
\end{frame}
\end{document}
