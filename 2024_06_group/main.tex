\documentclass{beamer}

\usepackage[utf8]{inputenc}
\usecolortheme{beaver}
\usepackage{caption}
\usepackage{subcaption}
\usepackage{mathtools}
\usepackage{todonotes}
\usepackage{amsmath}
\usepackage{bm}
\usepackage{listings}
\usepackage{ragged2e}
\usepackage{titlecaps}
\usepackage{fancyvrb}

\def\ci{\perp\!\!\!\!\!\perp}

\newtheorem{proposition}{Proposition}
\Addlcwords{for a is but and with of in as the etc on to if}

\setbeamertemplate{section in toc}{\inserttocsectionnumber.~\inserttocsection}
\usetheme{Boadilla}
\makeatletter
\setbeamertemplate{footline}{%
    \leavevmode%
    \hbox{%
        \begin{beamercolorbox}[wd=.3\paperwidth,ht=2.25ex,dp=1ex,center]{author in head/foot}%
            \usebeamerfont{author in head/foot}\insertshortauthor\expandafter\beamer@ifempty\expandafter{\beamer@shortinstitute}{}{~~(\insertshortinstitute)}
        \end{beamercolorbox}%
        \begin{beamercolorbox}[wd=.55\paperwidth,ht=2.25ex,dp=1ex,center]{title in head/foot}%
            \usebeamerfont{title in head/foot}\insertshorttitle
        \end{beamercolorbox}%
        \begin{beamercolorbox}[wd=.15\paperwidth,ht=2.25ex,dp=1ex,right]{date in head/foot}%
            \usebeamerfont{date in head/foot}\insertshortdate{}\hspace*{2em}
            \insertframenumber{} / \inserttotalframenumber\hspace*{2ex} 
        \end{beamercolorbox}}%
        \vskip0pt%
    }
\makeatother

\begin{document}

\title[]{A cautious approach to constraint-based causal model selection}
\date{}

\maketitle

\begin{frame}{Background: Causal DAGs and d-separation}
\end{frame}

\begin{frame}{Background: Example of constraint-based algorithm}
\end{frame}

\begin{frame}{Background: Epidemiological Setting}
	\begin{itemize}
		\item Moderate-dimensional setting, more than $ 4 $ variables but less than hundreds.
		\item Causal graph could be partially known.
		\item Sparsity should not be assumed.
	\end{itemize}
\end{frame}

\begin{frame}{Problems with the CI testing approach}
	\begin{itemize}
		\item Estimated Structures are usually too sparse, i.e., missing too many edges compared to the ground truth.
		\item Much statistical theory has been dedicated to controlling false positives in the sense of false edge inclusion such as multiple testing adjustments, limits on false discovery rate.
		\item In practice easy to achieve very few false edge inclusion.
		\item However, it is difficult to achieve low rates of error for false edge exclusions (high recall).
		\item Bias is largely driven by high rates of wrongly excluded edges.
	\end{itemize}
\end{frame}

\begin{frame}{Solution:Reformulation of the CI testing approach}
	\begin{itemize}
		\item Inst
	\end{itemize}
\end{frame}


\end{document}
